%!TEX Ts-program = lualatex
%!language = zh
\documentclass[main]{subfiles}
\ifSubfilesClassLoaded{\makeindex}{}
\begin{document}
\ifSubfilesClassLoaded{%
  \tableofcontents}{%
  \def\filename{section1}
}
\section{基本概念和例子}
\subsection{基本概念}
\subsubsection{随机过程的定义}
\begin{definition}\label{def:stochestic_progress}
  设\(I\)是非空指标集,\((\Omega,\mathcal{F},\mathbb{P})\)是概率空间。
  若\((X_\alpha:\alpha \in I)\)是一组定义在\((\Omega,\mathcal{F},\mathbb{P})\)上的随机变量(取值为\(\mathbb{R}^d\)),则称\((X_\alpha:\alpha \in I)\)为一个\index{sui ji guo cheng@随机过程|textbf}\textbf{随机过程}。
\end{definition}
\begin{definition}\label{def:independent_of_stochestic_progress}
  假设\((X_\alpha:\alpha \in I)\)和\((Y_\alpha:\alpha \in J)\)是两个随机过程。
  若对于任何有限序列\((s_1,\cdots,s_n) \subset I,(t_1,\cdots,t_m) \subset J\),都有\((X_{s_1},\cdots,X_{s_n}) \perp (Y_{t_1},\cdots,Y_{t_m})\),
  则称这两个\index{sui ji guo cheng du li@随机过程独立|textbf}\textbf{随机过程独立}。
\end{definition}
\subsubsection{轨道和修正}
\begin{definition}\label{def:path}
  设\((X_\alpha:\alpha \in I)\)为随机过程。固定\(\omega \in \Omega\),
  称\(t \mapsto X_{t}(\omega)\)为\(X\)的一条\index{gui dao@轨道|textbf}\textbf{轨道}。
\end{definition}
\begin{definition}\label{def:continous}
  称一个随机过程是(\index{zuo lian xu@左连续|textbf}\textbf{左连续}||\index{you lian xu@右连续|textbf}\textbf{右连续}||\index{lian xu@连续|textbf}\textbf{连续}||\index{zuo ji you lian@左极右连|textbf}\textbf{左极右连}||\index{zuo lian you ji@左连右极|textbf}\textbf{左连右极})的,若它的所有轨道都是(左连续||右连续||连续||左极右连||左连右极的)。
\end{definition}
\begin{definition}\label{def:same}
  设\((X_t:t \in I)\)和\((Y_t:t \in I)\)是两个随机过程。
  若\(\forall t \in I\),有\(\mathbb{P}(X_t=Y_t)=1\),则称它们互为\index{xiu zheng@修正|textbf}\textbf{修正}。
  若\(\mathbb{P}(\forall t \in I,X_t=Y_t)=1\),则称它们是\index{wu qu bie@无区别|textbf}\textbf{无区别}的。
\end{definition}
\begin{theorem}\label{the:1}
  设\((X_t:t \geq 0)\)和\((Y_t:t \geq 0)\)是两个右连续的随机过程,而\(D\)是\((0,\infty)\)的可数稠密子集。
  若\(\forall s \in D,\mathbb{P}(X_s=Y_s)=1\),则有\((X_t:t \geq 0)\)和\((Y_t:t \geq 0)\)是无区别的。
\end{theorem}
\subsubsection{有限维分布族}
为了简化记号,我们用\index{SI@\(S(I)\)|textbf}\(S(I)\)表示\(I\)的全体有序有限子集。
即:
\[
  S(I):=\{(t_1,\cdots,t_n):n \geq 1,t_i \in I,\forall i=1,\cdots,n\}
\]
用\(E\)表示\(\mathbb{R}^d\),用\(\mathcal{E}\)表示博雷尔代数。
\begin{definition}\label{def:finite_distribution_class}
  设\(I\)是非空指标集。若对于每个\(J \in S(I)\),都对应一个\((E^|J|,\mathcal{E}^|J|)\)上的概率测度\(u_J\),
  则称\((\mu_J:I \in S(I))\)为\(E\)上的一个\index{you xian wei fen bu zu@有限维分布族|textbf}\textbf{有限维分布族},其中每个\(\mu_J\)称为一个\index{you xian wei fen bu@有限维分布|textbf}\textbf{有限维分布}。
  设\(X=(X_t:t \in I)\)是一个随机过程,用\(\mu_{J}^X\)表示\((X_{t_1},\cdots,X_{t_n})\)的分布。
  称\(\mathcal{D}_{X}:=\{\mu_J^X:J \in S(I)\}\)为\(X\)的有限维分布族,称\(\mu_J^X\)为其中的一个有限维分布。
\end{definition}
\begin{definition}\label{def:equal}
  给定\((E,\mathcal{E})\)上的有限维分布族\(\mathcal{D}\),若存在随机过程\(X=(X_t:t \in I)\)使得\(\mathcal{D}_X=\mathcal{D}\),
  则称\(X\)为\(\mathcal{D}\)的一个\index{shi xian@实现|textbf}\textbf{实现}。
  若两个随机过程\(X,Y\)满足\(\mathcal{D}_X=\mathcal{D}_Y\),则称它们为\index{deng jia@等价|textbf}\textbf{等价}的。
  两个等价的过程互称实现。
  显然,两个互为修正的随机过程一定等价,反过来却未必。
\end{definition}
\subsubsection{左极右连实现}
\begin{definition}\label{def:zuojiyoulian}
  状态空间\(E=\mathbb{R}^d\)上的随机过程有\index{zuo ji you lian@左极右连}左极右连实现\(\iff\)它有左极右连修正。
  证明见教材p5
\end{definition}
\homework{1}
\subsection{随机游动}
\begin{definition}\label{def:svjiyzds}
  设\(\{\xi_n:n \geq 1\}\)是独立同分布的\(d\)维随机变量列,而\(X_0\)是与之独立的一个\(d\)维随机变量。
  令\(X_n:=X_0+\sum_{k=1}^{n} \xi_k\)。称\((X_n:n \geq 0)\)为\(d\)维\index{sui ji you dong@随机游动|textbf}\textbf{随机游动},并称\(\{\xi_n:n \geq 1\}\)为其\index{bu chang lie@步长列|textbf}\textbf{步长列}。
\end{definition}
\begin{definition}\label{def:jmdjsvjiyzds}
  若\(X_0,\xi_1\)均取值与\(\mathbb{Z}^d\),则该随机游动状态空间可以取为\(\mathbb{Z}^d\)。
  特别地,若还有\(\mathbb{P}(|\xi_1|=1)=1\),则称其为\index{jian dan sui ji you dong@简单随机游动|textbf}\textbf{简单随机游动}。
  进一步地,若对于\(\mathbb{Z}^d\)中的任一单位向量\(v\),均有\(\mathbb{P}(\xi_1=v)=\frac{1}{2d}\),则称其为\index{dui cheng jian dan sui ji you dong@对称简单随机游动|textbf}\textbf{对称简单随机游动}。
\end{definition}
\subsubsection{轨道的无界性}
方便起见,考虑\(\mathbb{Z}\)上的简单随机游动\(S_n\),设其步长列为\(\xi_n:n \geq 1\)。
设\(\mathbb{P}(\xi_n=1)=p,\mathbb{P}(\xi_n=-1)=q\),其中\(p,q \in (0,1),p+q=1\)。
\begin{theorem}\label{the:jihubirjwujp}
  \((S_n:n \geq 1)\)的轨道是几乎必然无界的。即:
  \begin{equation}
    \mathbb{P}(\sup_{n \geq 0}|S_n|=\infty)=1.
  \end{equation}
  证明见教材p9
\end{theorem}
\subsubsection{首达时分布}
\begin{definition}\label{def:Pi}
  记\index{pi@\(\mathbb{P}_i\)}\(\mathbb{P}_i(\cdot)=\mathbb{P}(\cdot\mid S_0=i)\)。
\end{definition}

\begin{definition}\label{def:uzdaui}
  定义\((S_n:n \geq 0)\)到达\(x \in \mathbb{Z}\)的\index{shou da shi@首达时|textbf}\textbf{首达时}\(\tau_x:=\inf \{n \geq 0:S_n=x\}\)。
\end{definition}
\begin{theorem}\label{the:122}
  当\(p=q=\frac{1}{2}\)时,对于\(a<b,i \in [a,b],a,b,i \in \mathbb{Z}\),有
  \begin{equation}
    \mathbb{P}_i(\tau_b<\tau_a)=\frac{i-a}{b-a},\mathbb{P}_i(\tau_a<\tau_b)=\frac{b-i}{b-a}
  \end{equation}
  当\(p \neq q\)时,有
  \begin{equation}
    \mathbb{P}_i(\tau_b<\tau_a)=\frac{1-(\frac{q}{p})^{i-a}}{1-(\frac{q}{p})^{b-a}},
    \mathbb{P}_i(\tau_a<\tau_b)=\frac{(\frac{q}{p})^{i-a}-(\frac{q}{p})^{b-a}}{1-(\frac{q}{p})^{b-a}}
  \end{equation}
  证明见教材p10
\end{theorem}
\begin{theorem}\label{the:1.2.3}
  当\(p \geq q\),对\(a \leq i \leq b \in \mathbb{Z}\),有
  \begin{equation}
    \mathbb{P}_i(\tau_a<\infty)=(\frac{q}{p})^{i-a},\mathbb{P}_i(\tau_b<\infty)=1
  \end{equation}
  当\(p \leq q\),有
  \begin{equation}
    \mathbb{P}_i(\tau_a<\infty)=1,\mathbb{P}_i(\tau_b<\infty)=(\frac{p}{q})^{b-i}
  \end{equation}
  证明见教材p11
\end{theorem}
\homework{2}
\subsection{布朗运动}
\subsubsection{背景和定义}
\begin{definition}\label{def:Brownian}
  假定\(\sigma^2 >0\),具有连续轨道的实值过程\((B_t: t \geq 0)\) 满足:
  \begin{enumerate}
    \item \(\forall 0 \leq s \leq t\), \(B_t-B_s \sim N(0,\sigma^2(t-s))\);
    \item \(\forall 0 \leq t_0 \leq \cdots \leq t_n\), \(B_0, B_1-B_0,\cdots,B_{t_n}-B_{t_{n-1}}\) 独立,
  \end{enumerate}
  称\((B_t:t \geq 0)\) 是以\(\sigma^2\)为参数的\index{bu lang yun dong@布朗运动|textbf}\textbf{布朗运动}。特别的,
  当\(\sigma^2=1\),\((B_t: t \geq 0)\)为\index{biao zhun bu lang yun dong@标准布朗运动|textbf}\textbf{标准布朗运动}。
\end{definition}
\begin{definition}\label{def:正态过程}
  有限维分布为正态分布的随机过程称为\index{zheng tai guo cheng@正态过程|textbf}\textbf{正态过程}。
\end{definition}
\subsubsection{布朗运动的构造}
\begin{theorem}\label{the:布朗运动存在}
  布朗运动是有连续实现的。
  证明见教材p13.
\end{theorem}
\subsubsection{布朗运动的性质}
\begin{theorem}\label{the:1.3.6}
  从原点出发的零均值高斯过程\((B_t: t \geq 0)\) 是标准布朗运动 \(\iff\) \(\forall s, t \geq 0\),
  \(\mathbb{E}(B_tB_s)=t \AND s\)。
  证明p17.
\end{theorem}
\begin{theorem}\label{the:1.3.7}
  布朗运动轨道几乎处处不可导。
  证明p17-18.
\end{theorem}
\homework{3}
\subsection{普瓦松过程}
\begin{definition}\label{def:Poisson Process}
  \((N_t: t \geq 0)\) 是非负整数不降随机过程,\(\alpha \geq 0\)满足:
  \begin{enumerate}
    \item \(\forall s,t \geq 0\), \(N_{s + t}-N_s \sim P(\alpha t)\),即:
      \(\mathbb{P}(N_{s + t}-N_s=k)=\frac{\alpha^kt^k}{k !}\mathrm{e}^{-\alpha t}\);
    \item \(\forall 0 \leq t_0 < \cdots< t_n\), \(N_0,N_{t_{1}}-N_{t_0},\cdots,N_{t_n}-N_{t_{n-1}}\)相互独立。
  \end{enumerate}
  称\((N_t:t \geq 0)\) 是\index{pu wa song guo cheng@普瓦松过程|textbf}\textbf{普瓦松过程},参数为\(\alpha\)。
\end{definition}
\subsubsection{跳跃间隔时间}
\((N_t: t \geq 0)\)以\(\alpha\)为参数的普瓦松过程,\(S_0=0,n \geq 1\), \(S_n=\inf \{t \geq 0: N_t-N_0 \geq 0\}\),\(\eta_n=S_n-S_{n-1}\)。
\(S_n\) 是\((N_t : t \geq 0)\)第\(n\)次跳跃等待时间,\(\eta_n\)第\(n-1\)次跳跃到第\(n\)次跳跃的间隔时间。
\begin{theorem}\label{the:1.4.1}
  \(\{\eta_n:n \geq 1\}\) 独立同分布,服从\(Exp(\alpha)\)。
  \(S_n, n \geq 1\),服从\(\Gamma(1,\alpha)\)。
  证明见P19.
\end{theorem}
\subsubsection{轨道重构}
\begin{theorem}\label{the:1.4.3}
  \(\{\eta_n:n \geq 1\}\) 独立同分布,服从\(Exp(\alpha),\gamma >0\)。
  \(S_0=0,S_n=\sum_{k=1}^{n} \eta_k\),则:
  \[
    N_t=\sum_{n=1}^{\infty} \mathbbm{1}_{S_n \leq t}=\sup\{n \geq 0: S_n \leq t\}.
  \]
  则随机过程\((N_t:t \geq 0)\)是以参数为\(\alpha\)的普瓦松过程。
\end{theorem}
\subsubsection{长时间极限行为}
\((N_t: t \geq 0)\)以\(\alpha\)为参数的普瓦松过程。
\begin{theorem}[\index{pu wa song guo cheng de qiang da shu ding lv@普瓦松过程的强大数定律}普瓦松过程的强大数定律]\label{the:普瓦松过程的强大数定律}
  \(\lim_{t \to \infty} \frac{N_t}{t} \overset{\text{a.s.}}{=} \alpha\)
  见p23.
\end{theorem}
\begin{theorem}[\index{pu wa song guo cheng de zhong xin ji xian ding li@普瓦松过程的中心极限定理}普瓦松过程的中心极限定理]\label{the:普瓦松过程的中心极限定理}
  \(\lim_{t \to \infty}\mathbb{P}(\frac{N_t-\alpha t}{\sqrt{\alpha t}} \leq x) = \frac{1}{\sqrt{2\pi}}\int_{\infty}^x \mathrm{e}^{-\frac{y^2}{2}}dy\)。
  见p23.
\end{theorem}
\begin{corollary}\label{cor:1.4.6}
  \(s,x >0\), \(\lim_{\lambda \to \infty}\mathrm{e}^{-\lambda s}\sum_{k \leq \lambda x}\frac{(\lambda s)^k}{k!}=\mathbbm{1}_{0<s<t}+\frac{1}{2}\mathbbm{1}_{\{s=x\}}\)。
  见p24.
\end{corollary}
\begin{theorem}[\index{la pu la si bian huan de fan yan gong shi@拉普拉斯变换的反演公式}拉普拉斯变换的反演公式]\label{the:拉普拉斯变换的反演公式}
  \(\xi\)是非负随机变量, \(L\)为其拉普拉斯变换,则\(\forall x >0\),
  \[
    \lim_{\lambda \to \infty}\sum_{k \leq \lambda x}\frac{(-\lambda)^k}{k!} \frac{\mathrm{d}^k}{\mathrm{d} \lambda^k}L(\lambda)=\mathbb{P}(\xi <x) + \frac{1}{2} \mathbb{P}(\xi=x)
  \]
  见p24.
\end{theorem}
\subsubsection{复合普瓦松过程}
\begin{definition}\label{def:复合普瓦松过程}
  \(\mu\)是\(\mathbb{R}\)上概率\(\mu(\{0\})=0\)。\((N_t: t \geq 0)\)以\(\alpha \geq 0\)为参数的零初值普瓦松过程,
  \(\{\xi_n:n \geq 1\}\)与\(N_t\)独立,具有相同分布\(\mu\),\(X_0\)与\((N_t),\{\xi_n\}\)独立。
  令:\(X_t=X_0 + \sum_{n=1}^{N_t}\xi_n, t \geq 0\),
  则\((X_t: t \geq 0)\)是以\(\alpha\)为跳跃速度,\(\mu\)为跳跃分布的\index{fu he pu wa song guo cheng@复合普瓦松过程|textbf}\textbf{复合普瓦松过程}。
\end{definition}
\begin{theorem}[\index{fu he pu wa song guo cheng de xing zhi@复合普瓦松过程的性质}复合普瓦松过程的性质]\label{the:复合普瓦松过程的性质}
  \((X_t:t \geq 0)\)为如上定义的复合普瓦松过程,则\index{fu he pu wa song guo cheng de xing zhi@复合普瓦松过程的性质}复合普瓦松过程的性质如下:
  \begin{enumerate}
    \item \(\forall s,t \geq 0, \theta \in \mathbb{R}\),\[
        \mathbb{E}\mathrm{e}^{\mathrm{i}\theta(X_{s + t}-X_s)}=\exp(\alpha t \int_{\mathbb{R}}(\mathrm{e}^{\mathrm{i}\theta x}-1)\mu(dx))\] ;
    \item \(\forall 0 \leq t_0 <\cdots<t_n, X_{t_0},X_{t_1}-X_{t_0},\cdots,X_{t_n}-X_{t_{n-1}} \) 相互独立。
  \end{enumerate}
\end{theorem}
\homework[4]
\subsection{普瓦松随机测度}
\subsubsection{定义和存在性}
\((E,\mathcal{E})\)为可测空间,\(\mu\)为\((E,\mathcal{E})\)上的\(\sigma\)有限测度。
\begin{definition}\label{def:整数值随机测度}
  \(\{X(B):B \in \mathcal{E}\}\)为取非负整数值随机过程,满足:
  \begin{enumerate}
    \item \(\forall B \in \mathcal{E}:\mu(B)<\infty\),则\(\mathbb{E}(X(B))=\mu(B)\)。
    \item \(\forall \{B_n:n \geq 1\} \in \mathcal{E}\)两两不交,
      则\(X(\bigcup_{k=1}^{\infty} B_k)=\sum_{k=1}^{\infty} X(B_k)\)。
  \end{enumerate}
  称\(\{X(B):B \in \mathcal{E}\}\)为以\(\mu\)为强度的\index{zheng shu zhi sui ji ce duo@整数值随机测度|textbf}\textbf{整数值随机测度}。
\end{definition}
\begin{definition}\label{def:普瓦松随机测度}
  \(\{X(B):B \in \mathcal{E}\}\)为整数值随机测度,满足:
  \begin{enumerate}
    \item \(\forall B \in \mathcal{E}:\mu(B)<\infty\),则\(X(B) \sim P(\mu(B))\),即:
      \(\mathbb{P}(X(B)=k)=\frac{\mu(B)^k}{k!}\mathrm{e}^{-\mu(B)},k=0,\cdots,n,\cdots\)。
    \item \(\forall \{B_n:n \geq 1\} \in \mathcal{E}\)两两不交,
      则\(\{X( B_k):k \in \mathbb{N}^+\}\)相互独立。
  \end{enumerate}
  称\(\{X(B):B \in \mathcal{E}\}\)为以\(\alpha\)为强度的\index{pu wa song sui ji ce duo@普瓦松随机测度|textbf}\textbf{普瓦松随机测度}。
\end{definition}
\begin{theorem}[\index{pu wa song sui ji ce duo de chong yao tiao jian@普瓦松随机测度的充要条件}普瓦松随机测度的充要条件]\label{the:普瓦松随机测度的充要条件}
  \(X\)为\((E,\mathcal{E})\)上以\(\mu\)为强度的整数值随机测度,则\(X\)为\index{pu wa song sui ji ce duo de chong yao tiao jian@普瓦松随机测度的充要条件}普瓦松随机测度的充要条件是
  \(\forall n \in \mathbb{N}^+,\xi_k \in \mathbb{R}, B_k \in \mathcal{E},k=1,\cdots,n\), \(B_i \cap B_j =\varnothing,i \neq j \),当\(\mu(B_k)< \infty,k=1,\cdots,n\),则\[
    \mathbb{E}\exp(\mathrm{i}\sum_{k=1}^{n} \theta_kX(B_k))=\exp(\sum_{k=1}^{n} (\mathrm{e}^{\mathrm{i}\theta_k}-1)\mu(B_k))
  \]
  见p28.
\end{theorem}
\begin{theorem}[\index{pu wa song sui ji ce duo de cun zai xing@普瓦松随机测度的存在性}普瓦松随机测度的存在性]\label{the:普瓦松随机测度的存在性}
  \(\mu\)为非零有限测度,\(\eta \sim P(\mu(E))\),\(\{\xi_k:k \in \mathbb{N}^+\}\) i.i.d.服从\(\mu(E)^{-1}\mu\)。\(\eta,\xi_1,\cdots,\xi_n,\cdots\)相互独立。
  令\(X=\sum_{j=1}^{\eta} \delta_{\xi_j}\),则\(X\)为以\(\mu\)为强度的普瓦松随机测度。
  见p29.
\end{theorem}
\begin{theorem}\label{the:1.5.5}
  \(\mu\)为\(\sigma\)有限测度,\(\{E_k:k \in \mathbb{N}^+\}\subset \mathcal{E},\mu(E_k)<\infty,k \in \mathbb{N}^+\),
  \(E=\bigcup_{k \in \mathbb{N}^+}E_k\),\(E_i \cap E_j=\varnothing,i \neq j\)。
  则存在\(X_k\)为\(E_k\)上的普瓦松随机测度强度为\(\mu_k:=\res{\mu}{E_k},k \in \mathbb{N}^+\)。
  令\(X=\sum_{j=1}^{\infty} X_j\),则\(X\)为以\(\mu\)为强度的普瓦松随机测度。
  见p29.
\end{theorem}
\subsubsection{积分与补偿的测度}
\begin{theorem}[\index{pu wa song sui ji ce duo de chong yao tiao jian a@普瓦松随机测度的充要条件2}普瓦松随机测度的充要条件2]\label{the:普瓦松随机测度的充要条件2}
  \(X\)为\((E,\mathcal{E})\)上以\(\mu\)为强度的整数值随机测度,则\(X\)为普瓦松随机测度的充要条件是
  \(\forall f \in \mathbb{R}^{\mathbb{R}}:\mu(f)< \infty ,\),则\[
    \mathbb{E}\exp(\mathrm{i}X(f))=\exp(\int_{E}(\mathrm{e}^{\mathrm{i}f(x)}-1)\mu(dx))
  \]
  见p28.
\end{theorem}
%%TODO:补偿测度???
\subsubsection{应用}
\begin{theorem}[\index{fu ge pu wa song guo cheng gou zao@复合普瓦松过程构造}复合普瓦松过程构造]\label{the:复合普瓦松过程构造}
  \(\nu\)为\(\mathbb{R}\)上非零有限测度。\(N(ds,dz)\)为\((0,\infty)\times \mathbb{R}\)上以\(ds \nu(dz)\)为强度的普瓦松随机测度,
  \(ds\)为勒贝格测度,\(X_0\)与\(N(ds,dz)\)独立,\[
    X_t=X_0 + \iint_{(0,t]\times \mathbb{R}}z N(ds,dz),t \geq 0
  \]
  则\((X_t:t \geq 0)\)具有跳跃速率\(\alpha\)和跳跃分布\(\mu:=\nu(\mathbb{R})^{-1}\nu\)的复合普瓦松过程。
\end{theorem}
\begin{theorem}[\index{fu ge pu wa song guo cheng gou zao@复合普瓦松过程构造2}复合普瓦松过程构造2]\label{the:复合普瓦松过程构造2}
  \(\mu\)为\(\mathbb{R}\)上非零\(\sigma\)有限测度,\(\mu(\{0\})=0\),\(q\)为\(\mathbb{R}\)上非负博雷尔可测函数,\(0 < \beta :=\mu(q)<\infty\)。
  \(N(ds,dz,du)\)为\((0,\infty)\times \mathbb{R}\times(0,\infty)\)上以\(ds \mu(dz)du\)为强度的普瓦松随机测度,
  \(ds\)为勒贝格测度,\(X_0\)与\(N(ds,dz)\)独立,\[
    X_t=X_0 + \iiint_{(0,t]\times \mathbb{R}\times[0,q(z)]}z N(ds,dz,du),t \geq 0
  \]
  则\((X_t:t \geq 0)\)具有跳跃速率\(\beta\)和跳跃分布\(\beta^{-1}q(z)\mu(dz)\)的复合普瓦松过程。
\end{theorem}
\homework[5]

\ifSubfilesClassLoaded{%
  \printindex }{%
}

\end{document}
