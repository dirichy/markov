\documentclass[../main]{subfiles}
\begin{document}
\begin{lemma}\label{lem:1}
  Assume \((B_t:t \geq 0)\) is a random process ranging in \(\mathbb{R}\), \(a \in \mathbb{R}^+\), and \(\forall s,t:0 \leq s \leq t,B_t-B_s \sim N(0,a(t-s))\).
  Assume \(B_t\) is continuous about \(t\), a.s.
  Let \(\mathcal{F}_t:=\sigma(B_s:0 \leq s \leq t)\).
  Then \((B_t:t \geq 0)\) is Brownian motion \(\iff \forall 0 \leq s \leq t,B_t - B_s \perp \mathcal{F}_s\).
\end{lemma}
\begin{proof}
  ``\(\implies\)'': To prove \(B_t-B_s \perp \mathcal{F}_s\), only need to prove for \(t_1 < t_2 < \cdots <t_{n-1}=s<t=t_n\), we have
  \(B_t-B_s \perp \sigma(B_{t_k}:k=1,\cdots,n-1)\).
  Easily \(B_t-B_s \perp \sigma(B_{t_{k+1}}-B_{t_k},B_{t_1}:k=1,\cdots,n-2)=\sigma(B_{t_k}:k=1,\cdots,n-1)\), so \(B_t-B_s \perp \mathcal{F}_s\).

  ``\(\impliedby\)'': For \(t_1 < \cdots < t_n\), we need to prove \(B_{t_1},B_{t_{k+1}}-B_{t_k}:1 \leq k \leq n-1\) are independent.
  Use MI to \(n\). When \(n=1\) it's obvious.
  Assume we have proved it for certain \(n \geq 1\), now consider \(n+1\).
  Since \(B_{t_{k+1}}-B_{t_k} \in \mathcal{F}_{t_n},k=1,\cdots,n-1\), we have \(B_{t_{n+1}}-B_{t_n} \perp \sigma(B_{t_1},B_{t_{k+1}}-B_{t_k}:k=1,\cdots,n-1)\).
  So \(\mathbb{P}(B_{t_1} \in A_1,B_{t_{k+1}}-B_{t_k} \in A_{k+1},k=1,\cdots,n)=\mathbb{P}(B_{t_1} \in A_1,B_{t_{k+1}}-B_{t_k} \in A_{k+1},k=1,\cdots,n-1)\mathbb{P}(B_{t_{n+1}}-B_{t_n} \in A_{n+1})\).
  By Induction assumption we get \(\mathbb{P}(B_{t_1} \in A_1,B_{t_{k+1}}-B_{t_k} \in A_{k+1},k=1,\cdots,n)=\mathbb{P}(B_{t_1}\in A_1) \prod_{k=1}^{n}\mathbb{P}(B_{t_{k+1}}-B_{t_k} \in A_{k+1})\).
  So finally we get \(B_{t_1},B_{t_{k+1}}-B_{t_k}:1 \leq k \leq n\) are independent.
\end{proof}

\begin{problem}\label{pro:1}
  Assume \((B_t:t \geq 0)\) is Brownian motion, prove that for \(r>0\), we have \((B_{t + r}-B_r:t \geq 0)\) is Brownian motion, too.
\end{problem}
\begin{solution}
  Assume \(B_t-B_s \sim N(0,a(t-s)),a >0\).
  Let \(\mathcal{F}_t:=\sigma(B_s:0 \leq s \leq t)\) and \(\mathcal{G}_t:=\sigma(B_{r+s}-B_r:0 \leq s \leq t)\).
  For \(0 \leq s \leq t\), we have \(B_{t+r}-B_r -(B_{s+r}-B_r)=B_{t+r}-B_{s+r} \sim N(0,a(t-s))\).
  And easily to know \(B_{r + s}-B_r \in \mathcal{F}_{r+s}\), so \(\mathcal{G}_t \subset \mathcal{F}_{t+r},\forall t \geq 0\).
  Since \((B_t:t \geq 0)\) is Brownian motion, easily \(\mathcal{F}_{s+r} \perp B_{t+r}-B_{s+r}\).
  Since \(\mathcal{G}_{t}\subset \mathcal{F}_{t +r}\), we obtain \(\mathcal{G}_t \perp B_{t + r}-B_{s + r}=B_{t+r}-B_r -(B_{s+r}-B_r)\).
  Easily since \(B_t\) is continuous we get \(B_{t + r}-B_r\) is continuous.
  So \((B_{t+r}:t \geq 0)\) is Brownian motion.
\end{solution}

\begin{problem}\label{pro:2}
  Assume \((B_t:t \geq 0)\) is standard Brownian motion start at \(0\).
  Prove that \(\forall c>0,(c B_{\frac{t}{c^2}}:t \geq 0)\) is standard Brownian motion start at \(0\), too.
\end{problem}
\begin{solution}
  Since \(B_0=0\) we get \(c B_{\frac{0}{c^2}}=0\).
  Let \(\mathcal{F}_t:=\sigma(B_s:0 \leq s \leq t)\) and \(\mathcal{G}_t:=\sigma(c B_{\frac{s}{c^2}}:0 \leq s \leq t)\).
  Easily to know \(\mathcal{G}_t=\mathcal{F}_{\frac{t}{c^2}}\).
  For \(0 \leq s \leq t\), we have \(c B_{\frac{t}{c^2}}-c B_{\frac{s}{c^2}}= c(B_{\frac{t}{c^2}}-B_{\frac{s}{c^2}}) \sim N(0,t-s)\),
  because \(B_{\frac{t}{c^2}}-B_{\frac{s}{c^2}} \sim N(0,\frac{t-s}{c^2})\).
  And since \((B_t:t \geq 0)\) is Brownian motion, we get \(B_{\frac{t}{c^2}}-B_{\frac{s}{c^2}} \perp \mathcal{F}_{\frac{s}{c^2}}=\mathcal{G}_s\).
  Easily since \(B_t\) is continuous we get \(c B_{t_c^2}\) is continuous.
  So \((c B_{\frac{t}{c^2}}:t \geq 0)\) is standard Brownian motion starts at \(0\), too.
\end{solution}
\begin{problem}\label{pro:3}
  Assume \((X_t:t \geq 0)\) and \((Y_t:t \geq 0)\) are two independent standard Brownian motion, \(a,b \in \mathbb{R}\) and \(\sqrt{a^2 + b^2} >0\).
  Prove that \((aX_t + bY_t:t \geq 0)\) is a Brownian motion with parameter \(c^2=a^2 + b^2\).
\end{problem}
\begin{solution}
  Let \(\mathcal{F}_t:=\sigma(X_s:0 \leq s \leq t)\) and \(\mathcal{G}_t:=\sigma(Y_s:0 \leq s \leq t)\).
  Let \(\mathcal{H}_t:=\sigma(a X_s + b Y_s:0 \leq s \leq t)\).
  Since \((X_t:t \geq 0),(Y_t:t \geq 0)\) are two independent Brownian motion, we know \(\forall 0 \leq s \leq t,X_t - X_s \perp \mathcal{F}_s,\mathcal{G}_s;Y_t-Y_s \perp \mathcal{F}_s,\mathcal{G}_s\).
  So we get \(aX_t+bY_t-aX_s-bY_s \perp \mathcal{F}_s,\mathcal{G}_s\), thus \(aX_t+bY_t-aX_s-bY_s \perp \sigma(\mathcal{F}_s,\mathcal{G}_s)\)
  Easily \(a X_s+bY_s \in \sigma(\mathcal{F}_s,\mathcal{G}_s)\), so \(\mathcal{H}_t \subset \sigma(\mathcal{F}_t,\mathcal{G},t),\forall t \geq 0\).
  So \(a X_t+bY_t-aX_s-bY_s \perp \mathcal{H}_s\).
  And easily \(a(X_t-X_s) \sim N(0,a^2(t-s)),b(Y_t-Y_s) \sin N(0,b^2(t-s))\), and since \(\mathcal{F}_t \perp \mathcal{G}_t\) we get \(a(X_t-X_s) \perp b(Y_t-Y_s)\),
  so \(a X_t+bY_t-aX_s-bY_s \sim N(0,(a^2 + b^2)(t-s))\).
  Easily since \(X_t,Y_t\) is continuous we get \(a X_t + b Y_t\) is continuous.
  So \((a X_t + b Y_t:t \geq 0)\) is a Brownian motion with parameter \(a^2 + b^2 = c^2\).
\end{solution}
\begin{problem}\label{pro:4}
  Assume \((B_t:t \geq 0)\) is standard Brownian motion start at \(0\).
  Let \(X_0=0\) and \(X_t:=t B_{\frac{1}{t}}\).
  Given
  \[
    \limsup_{t \to \infty}\frac{B_{t}}{\sqrt{2t \log \log t}}=1
  \]
  Prove that \((X_t:t \geq 0)\) is standard Brownian motion start at \(0\).
\end{problem}
\begin{solution}
  First consider the distribution of \(X_t-X_s\) for \(s <t\).
  If \(s=0\) then \(X_t-X_s=t B_{\frac{1}{t}} \sim N(0,t^2 \times \frac{1}{t})=N(0,t)\).
  Else, \(s >0\), then \(X_t-X_s =s(B_{\frac{1}{t}}-B_{\frac{1}{s}})+(t-s)B_{\frac{1}{t}}\).
  Easily \(B_{\frac{1}{s}}-B_{\frac{1}{t}} \sim N(0,\frac{1}{s})-\frac{1}{t})\),
  and \(B_{\frac{1}{t}} \sim N(0,\frac{1}{t})\), and since \(\frac{1}{t}<\frac{1}{s}\) we know
  \(B_{\frac{1}{s}}-B_{\frac{1}{t}}\perp B_{\frac{1}{t}}\).
  So \(X_t-X_s \sim N(0,s^2(\frac{1}{s}-\frac{1}{t})+(t-s)^2 \frac{1}{t})=N(0,t-s)\).

  Second let \(\mathcal{G}_t:=\sigma(X_s:0 \leq s \leq t)\),
  we need to check \(X_t-X_s \perp \mathcal{G}_s,\forall 0 \leq s \leq t\).
  For \(s=0\) we get \(\mathcal{G}_s=\{\emptyset,\Omega\}\), so it's obvious.
  Now assume \(s>0\). Then \(\mathcal{G}_s=\sigma(B_{\frac{1}{r}}:0 \leq r \leq s)\).
  Only need to prove for any finite set \(I \subset [0,s]\), we have \(X_t-X_s \perp \sigma(B_{\frac{1}{r}}:r \in I)\).
  Only need to check \(\mathbb{E}(B_{\frac{1}{r}}(X_t-X_s))=0\) because they are all normal distributed random variable.
  Easily \(\mathbb{E}(B_{\frac{1}{r}}X_t)=\mathbb{E}((B_{\frac{1}{r}}-B_{\frac{1}{t}})tB_{\frac{1}{t}})+\mathbb{E}(t B_{\frac{1}{t}}^2)=1\),
  and \(\mathbb{E}(B_{\frac{1}{r}}X_s)=\mathbb{E}((B_{\frac{1}{r}}-B_{\frac{1}{s}})sB_{\frac{1}{s}})+\mathbb{E}(s B_{\frac{1}{s}}^2)=1\)
  So \(\mathbb{E}(B_{\frac{1}{r}}(X_t-X_s))=0\).

  Finally we need to check \(X_t\) is continuous a.s.
  Easily for \(t \neq 0\) we know \(X_t\) is continuous at \(t\).
  Only need to check \(X_t\) is continuous at \(0\) with probability \(1\).
  Easily to know \((-B_t:t \geq 0)\) is standard Brownian motion, too.
  So \(\limsup_{t \to \infty}\frac{-B_t}{\sqrt{2t \log \log t}}=1\).
  So \(\limsup_{t \to \infty}\frac{|B_t|}{2t \log \log t}=1\).
  So \(\limsup_{t \to 0+}|X_t|=\lim_{t \to \infty} |\frac{1}{t}B_{t}| \leq \limsup_{t \to \infty}\frac{\sqrt{2 t \log \log t}}{t}=0\).
  So \(\lim_{t \to 0+}|X_t|=0\).
  So \(X_t\) is continuous with probability \(1\).
\end{solution}
\end{document}
